
%----------------------------------------------------------------------------------------
%    INTRODUCTION
%----------------------------------------------------------------------------------------

\cleardoublepage
\chapter{Introduction} % The asterisk leaves out this chapter from the table of contents
\label{ch:introduction}

\section{Language Conventions}\index{language conventions}

\newthought{Language conventions} are sometimes different between physicists and statisticians.

\begin{table} % Add the following just after the closing bracket on this line to specify a position for the table on the page: [h], [t], [b] or [p] - these mean: here, top, bottom and on a separate page, respectively
    \centering % Centers the table on the page, comment out to left-justify
    \begin{tabular}{l l} % The final bracket specifies the number of columns in the table along with left and right borders which are specified using vertical bars (|); each column can be left, right or center-justified using l, r or c. To specify a precise width, use p{width}, e.g. p{5cm}
        \toprule % Top horizontal line
        Physicists say & Statisticians say \\ % Column names row
        \midrule % In-table horizontal line
        determine      & estimate          \\ % Content row 1
        estimate       & guess             \\ % Content row 2
        \bottomrule % Bottom horizontal line
    \end{tabular}
    \caption{Language conventions: example 1} % Table caption, can be commented out if no caption is required
    \label{tab:language-conventions-example-1} % A label for referencing this table elsewhere, references are used in text as \ref{label}
\end{table}

\begin{table}
    \centering
    \begin{tabular}{l l}
        \toprule
        Demographic language & Physics language\\
        \midrule
        Sample & Data (set)\\
        Draw a sample & Observe, measure\\
        Sample of size $N$ & $N$ observations \\
        Population & Observable space\\
        \bottomrule
    \end{tabular}
    \caption{Language conventions: example 2}
    \label{tab:language-conventions-example-2}
\end{table}

\begin{itemize}[$\to$]
    \item Think about a census, or an election poll.
        \marginnote{We will specify: ``parent mean'' = population mean = mean of the underlying distribution; or: ``sample mean'' refers to the empirical average.}
    \item We need to distinguish between the properties of the sample and those of the underlying population.
\end{itemize}

We will avoid \hlred{misleading terms}\index{language conventions!misleading terms}:

\begin{itemize}
    \item error $\to$ variance, confidence interval, interval estimate, credible interval
    \item propagation of error $\to$ change of variable
\end{itemize}

\section{Two Philosophies}\index{two philosophies}

\newthought{Two philosophies} are commonly distinguished: Bayesian and frequentist (or classical).

\begin{itemize}[$\to$]
    \item \newthought{Bayesian} approach\index{Bayesian approach} measures the ``degree of belief''\index{degree of belief} that a statement is true.
    \item \newthought{Frequentist} approach\index{frequentist approach} measures the relative frequency of something happening.
\end{itemize}

\begin{table}
    \centering
    \begin{tabular}{l c c}
        \toprule
        Property                                    & Bayesian & Frequentist \\
        \midrule
        Statement is observer-dependent             & \true    & \false      \\
        Applicable to repeatable cases              & \true    & \true       \\
        Applicable to future unknown facts          & \true    & \false      \\
        Applicable to past unknown facts            & \true    & \false      \\
        \tabincell{l}{Applicable to possible outcomes \\
            \quad of an experiment}                 & \true    & \true       \\
        Applicable to the true value of a parameter & \true    & \false      \\
        \tabincell{l}{Allows goodness-of-fit \\
            \quad (single-hypothesis testing)}      & \false   & \true       \\
        Allows decision theory                      & \true    & \false      \\
        \bottomrule
    \end{tabular}
    \caption{Comparison between Bayesian and frequentist philosophies}
    \label{tab:two-philosophies}
\end{table}

\section{Notation} \index{language conventions!notation}

\begin{itemize}[$\to$]
    \item \mono{Greek letters}: parameters of the theory, \eg $\theta$, $\mu$, $\sigma$, $\cdots$
    \item \mono{Roman letters}: random variables corresponding to physical observables, \eg $x$, $E$, $\cdots$
    \item \mono{Capital letters}: used for probability distributions, \eg $P$\index{probability distribution}; and cumulative distributions, \eg $F$\index{cumulative distribution}
    \item \mono{Lowercase letters}: used for probability density functions, \eg $p$ or $f$\index{probability density function}
    \item \mono{Bar notation}: average values, \eg $\bar{x}$, $\bar{E}$, $\cdots$
    \item \mono{Hat notation}: estimates of parameters (often the mode\index{mode} of the parameter distribution), \eg $\hat{\theta}$, $\hat{\mu}$, $\hat{\sigma}$, $\cdots$
\end{itemize}
