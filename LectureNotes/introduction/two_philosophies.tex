
%------------------------------------------------

\section{Two Philosophies}\index{two philosophies}

\newthought{Two philosophies} are commonly distinguished: Bayesian and frequentist (or classical).

\begin{itemize}[$\to$]
    \item \newthought{Bayesian} approach\index{Bayesian approach} measures the ``degree of belief''\index{degree of belief} that a statement is true.
    \item \newthought{Frequentist} approach\index{frequentist approach} measures the relative frequency of something happening.
\end{itemize}

\begin{table}
    \centering
    \begin{tabular}{l c c}
        \toprule
        Property                                    & Bayesian & Frequentist \\
        \midrule
        Statement is observer-dependent             & \true    & \false      \\
        Applicable to repeatable cases              & \true    & \true       \\
        Applicable to future unknown facts          & \true    & \false      \\
        Applicable to past unknown facts            & \true    & \false      \\
        \tabincell{l}{Applicable to possible outcomes \\
            \quad of an experiment}                 & \true    & \true       \\
        Applicable to the true value of a parameter & \true    & \false      \\
        \tabincell{l}{Allows goodness-of-fit \\
            \quad (single-hypothesis testing)}      & \false   & \true       \\
        Allows decision theory                      & \true    & \false      \\
        \bottomrule
    \end{tabular}
    \caption{Comparison between Bayesian and frequentist philosophies}
    \label{tab:two-philosophies}
\end{table}
