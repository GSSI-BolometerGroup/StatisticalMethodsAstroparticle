\section{Language Conventions}\index{language conventions}

\newthought{Language conventions} are sometimes different between physicists and statisticians.

\begin{table} % Add the following just after the closing bracket on this line to specify a position for the table on the page: [h], [t], [b] or [p] - these mean: here, top, bottom and on a separate page, respectively
    \centering % Centers the table on the page, comment out to left-justify
    \begin{tabular}{l l} % The final bracket specifies the number of columns in the table along with left and right borders which are specified using vertical bars (|); each column can be left, right or center-justified using l, r or c. To specify a precise width, use p{width}, e.g. p{5cm}
        \toprule % Top horizontal line
        Physicists say & Statisticians say \\ % Column names row
        \midrule % In-table horizontal line
        determine      & estimate          \\ % Content row 1
        estimate       & guess             \\ % Content row 2
        \bottomrule % Bottom horizontal line
    \end{tabular}
    \caption{Language conventions: example 1} % Table caption, can be commented out if no caption is required
    \label{tab:language-conventions-example-1} % A label for referencing this table elsewhere, references are used in text as \ref{label}
\end{table}

\begin{table}
    \centering
    \begin{tabular}{l l}
        \toprule
        Demographic language & Physics language\\
        \midrule
        Sample & Data (set)\\
        Draw a sample & Observe, measure\\
        Sample of size $N$ & $N$ observations \\
        Population & Observable space\\
        \bottomrule
    \end{tabular}
    \caption{Language conventions: example 2}
    \label{tab:language-conventions-example-2}
\end{table}

\begin{itemize}[$\to$]
    \item Think about a census, or an election poll.
    \marginnote{We will specify: ``parent mean'' = population mean = mean of the underlying distribution; or: ``sample mean'' refers to the empirical average.}
    \item We need to distinguish between the properties of the sample and those of the underlying population.
\end{itemize}

We will avoid \hlred{misleading terms}\index{language conventions!misleading terms}:

\begin{itemize}
    \item error $\to$ variance, confidence interval, interval estimate, credible interval
    \item propagation of error $\to$ change of variable
\end{itemize}
