
%------------------------------------------------

\section{Definition of point estimation}\index{point estimation}
\label{sec:def_of_point_estimation}

Most of this is frequentist exclusively, even if some methods are easily applicable to a Bayesian approach.

\subsection{Inference}\index{inference}
\label{subsec:point_estimation_inference}

The inference is the process of determining an estimated value $\hat{\theta}$ and the corresponding uncertainty of some parameter $\theta$ from experimental data.

$$
	\mathrm{Theory \ model} 
	\overset{\mathrm{probability}}
	{
		\underset{\mathrm{inference}}
		{\rightleftharpoons}
	} 
	  \mathrm{data}
$$

\begin{itemize}
	\item Probability: data fluctuate according to randomness of process and of the experimental response.
	\item Inference: model parameters are uncertain due to fluctuations in the finite data sample.
\end{itemize}

\subsection{Estimators and estimates}\index{estimator}\index{estimate}
\label{subsec:point_estimator_and_estimate}

The estimate of an unknown parameter is a mathematical procedure to determine the central value of the parameter as a function of the observed data sample.

The function of the data sample that returns the estimate is called “estimator”.

\newthought{Example}: If I measure the mass of an object once, the measured value is an estimate of the object mass:

$$
\hat{m}(x) = x
$$

\paragraph{Properties of estimators}

\begin{itemize}[$\to$]
	\item Consistency
	\item Unbiasedness
	\item Information content or efficiency
	\item Robustness
\end{itemize}