
%------------------------------------------------

\section{Choose approaches and specify the questions}
\label{sec:measurement_approach}

So far, we have used a very vague language on purpose. To be more specific, we need to choose either the frequentist or the Bayesian approach, and specify the questions addressed by each of them.

\subsection{Frequentist approach}\index{frequentist approach}
\label{subsec:measurement_approach_frequentist}

\paragraph{Assumptions}

The true value of the parameter $\theta$ is fixed but unknown. We cannot associate a PDF to $\theta$, but just to the data $\vec{x}$.

\paragraph{Point estimation}

Based on the measured data, what is our best “estimate” for the fixed unknown parameter?

What is the estimate that is closer to the true value?

\paragraph{Interval estimation}

\marginnote[6pt]{If we repeat the measurement 100 times, we will have 100 different intervals, and the true value will be contained in them 68 times.}

Based on the measured data, what interval contains the true value with a predefined amount of probability (\eg 68\%)?

\paragraph{Goodness of fit}

Does my model provide a suitable description of the data, or is there any indication that it should be modified \mono{somehow}?

\paragraph{Hypothesis testing}

\marginnote[6pt]{Assuming $H_{0}$ is true, what is the probability that the data will fake $H_{1}$ (and viceversa)?}

Based on the data, which among two (or more) alternative hypotheses is true?

\subsection{Bayesian approach}\index{Bayesian approach}
\label{subsec:measurement_approach_bayesian}

In the Bayesian approach, the probability is interpreted as a “degree of belief”\index{degree of belief} and can be therefore applied to a wider range of elements, including:

\begin{itemize}
	\item random variables
	\item (true) parameters of a model
	\item hypotheses
\end{itemize}

\paragraph{Point estimation}

Based on the measured data, what is the most probable value for the parameter $\theta$?

\paragraph{Interval estimation}

Based on the measured data, what is the interval of the PDF of $\theta$, $f(\theta)$, that contains a given amount of probability (\eg 68\%)?

\paragraph{Goodness of fit}

This question makes no sense in the Bayesian approach, because we cannot compare one hypothesis with $n$ unknown ones.

\paragraph{Hypothesis testing}

\marginnote[6pt]{Assuming $H_{0}$ is true, what is the probability that the data will fake $H_{1}$ (and viceversa)?}

Based on the data, what is the ratio of the probabilities of hypotheses $H_{0}$ and $H_{1}$?
